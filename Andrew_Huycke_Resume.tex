\documentclass{resume} % Use the custom resume.cls style

\usepackage{enumitem}
\usepackage[left=0.5 in,top=0.5in,right=0.5 in,bottom=0.5in]{geometry} % Document margins
\newcommand{\tab}[1]{\hspace{.2667\textwidth}\rlap{#1}} 
\newcommand{\itab}[1]{\hspace{0em}\rlap{#1}}
\name{Andrew Huycke} % Your name
% You can merge both of these into a single line, if you do not have a website.
\address{918-800-9785 \\ ahuycke3@gatech.edu \\ github.com/ahuycke \\ https://www.linkedin.com/in/andrew-huycke}
%\address{\href{mailto:contact@faangpath.com}{contact@faangpath.com} \\ \href{https://linkedin.com/company/faangpath}{linkedin.com/company/faangpath} \\ \href{www.faangpath.com}{www.faangpath.com}}  %

\begin{document}
\vspace{-10pt}
%----------------------------------------------------------------------------------------
%	EDUCATION SECTION
%----------------------------------------------------------------------------------------

\begin{rSection}{Education}

{\bf Georgia Institute of Technology} \hfill {Aug 2024 - May 2025}\\
Master of Science in Analytics (In Person) \\
{\bf Colorado School of Mines} \hfill {Aug 2021 - May 2024}\\
Bachelor's of Science in Computer Science + Data Science \hfill {GPA: 3.99}\\
\textbf{Relevant Coursework}: Machine Learning, Data Structures, Algorithms, Linux OS, Data Science, Software Engineering, Database Management, Multivariate Analysis, Applied Statistics, Probability, Linear Algebra

%Minor in Linguistics \smallskip \\
%Member of Eta Kappa Nu \\
%Member of Upsilon Pi Epsilon \\


\end{rSection}
\vspace{-6pt}

\begin{rSection}{EXPERIENCE}

\textbf{Nuveen} \hfill Jun - Aug 2024\\
Data Science and Technology Intern \hfill \textit{Charlotte, NC}
\vspace{-6pt}
 \begin{itemize}[leftmargin=*]
    \itemsep -6pt {} 
    \item Automated business metric extraction for the Responsible Investments Team, cutting data processing time per document from 12 minutes to 10 seconds.
    \item Built backend with \textbf{Python} and \textbf{Langchain} in order to interface with large language models
    \item Implemented retrieval augmented generation (RAG) to improve quality of extracted data
    \item Deployed the application in an elastic cloud compute server in \textbf{AWS} for scalability
 \end{itemize}

 \textbf{HiLabs} \hfill Aug - Dec 2023\\
Field Session Intern \hfill \textit{Golden, CO}
\vspace{-6pt}
 \begin{itemize}[leftmargin=*]
    \itemsep -6pt {} 
    \item Created a large language model (LLM) based educational tool to assess student knowledge in courses
    %\item Utilized \textbf{Langchain} and \textbf{llama} within \textbf{Python} for LLM prompting
    \item Built knowledge graphs of materials with \textbf{NetworkX}, leading to an 80\% decrease in question generation time
    \item Performed iterative unit testing on LLM answer validation prompts, resulting in a 60\% increase in accuracy
 \end{itemize}

\textbf{RSM} \hfill Jun - Aug 2023\\
Product and Strategy Intern \hfill \textit{Greenwood Village, CO}
\vspace{-6pt}
 \begin{itemize}[leftmargin=*]
    \itemsep -6pt {} 
     \item Designed and assembled a database of client information in \textbf{SQL} to facilitate client-specific recommendations 
     \item Built a \textbf{Python} script to automate the data entry pipeline for all employees
     \item Leveraged \textbf{x++} to scrape client data from Dynamics 365
     \item Developed a \textbf{PowerApp} to allow for seamless viewing and querying of the client database
 \end{itemize}

\end{rSection}
%\vspace{-4pt}
 
%\vspace{-4pt}

%
%\vspace{-4pt}
%
%\textbf{Math Department Grader} \hfill Jan 2022 - May 2022%\\
%Colorado School of Mines \hfill \textit{Golden, CO}
%\vspace{-6pt}
 %\begin{itemize}
%    \itemsep -6pt {} 
%     \item Responsible for grading all assignments for a section of MATH 225: Differential Equations
% \end{itemize}

%\end{rSection} 

%\vspace{-6pt}
%----------------------------------------------------------------------------------------
%	PROJECTS
%----------------------------------------------------------------------------------------

\begin{rSection}{PROJECTS}

\textbf{AI Chef Project (1st place out of 12 groups in CSCI 470)} \hfill %\\
%Colorado School of Mines \hfill \textit{Golden, CO}
\vspace{-6pt}
 \begin{itemize}[leftmargin=*]
    \itemsep -6pt {} 
     \item Leveraged the \textbf{OpenAI API} to generate customized recipes for users based on ingredients available, dietary restrictions, price restrictions, and more
     \item Built user interface on top of \textbf{Streamlit} to allow for user input, image display, and recipe display
 \end{itemize}

%\vspace{-1.25em}
\textbf{NBA Player Height, Weight, and Position Predictor} %\hfill 2022
\vspace{-6pt}
 \begin{itemize}[leftmargin=*]
    \itemsep -6pt {} 
     \item Created machine learning models in \textbf{Python} using \textbf{Sklearn, Pandas, and Numpy} to predict NBA player heights, weights, and positions given their stats, with the best model achieving 0.81 accuracy
     \item Leveraged \textbf{BeautifulSoup} and \textbf{Request} libraries in Python to scrape player information
 \end{itemize}

 %\vspace{-4pt}

%
% \textbf{TA/Mentor for Introduction to Computer Science} \hfill Aug 2022 - Dec 2023%\\
%Colorado School of Mines \hfill \textit{Golden, CO}
%\vspace{-6pt}
% \begin{itemize}
%    \itemsep -6pt {} 
%     \item Assisted students with Python coding projects during office hours and in class
%     \item Organized meetings with 20-30 students each semester to provide course guidance
% \end{itemize}
 
%\item \textbf{Lagrange Interpolation/Polynomial Regression Application} %\hfill 2022
%\vspace{-6pt}
% \begin{itemize}
%    \itemsep -6pt {} 
%     \item Developed an application from scratch in \textbf{C++} that computes the lagrange interpolating polynomial of data
%     \item Incorporated ordinary least squares regression to give polynomial regression curves of any appropriate order
%     \item Integrated graphics through the \textbf{SFML} library used to create a plot of points and regression curves
%    \item Applied topics from linear algebra to build a matrix class with necessary methods. Topics include computing matrix determinants recursively, computing adjugate matrices, and computing the inverse of a matrix
% \end{itemize}

 %\vspace{-4pt}
 
%\item \textbf{Bottle Rocket Motion Modeling} \hfill Fall 2022
%\vspace{-6pt}
% \begin{itemize}
%    \itemsep -6pt {} 
%     \item Created interactive \textbf{Python} program that models the motion of bottle rockets that have varying initial pressures, water volumes, coefficients of drag, and masses
%     \item Incorporated Euler's method to calculate thrust and drag forces, as well as the acceleration, velocity, and position
%     \item Utilized the \textbf{Matplotlib} library to create visual summaries of the bottle rocket flight
 %\end{itemize}
 
\end{rSection} 
\vspace{-6pt}
%----------------------------------------------------------------------------------------
% TECHINICAL STRENGTHS	
%----------------------------------------------------------------------------------------
\begin{rSection}{TECHNICAL SKILLS}
%\begin{tabular}{ @{} >{\textit}l @{\hspace{6ex}} l }
\textbf{Languages:} Python, R, C++, C, Java, PostgreSQL, HTML, CSS, JavaScript, Bash, RISC-V, LaTeX, x++\\
\textbf{Development Tools:} VS Code, Linux, Git, Jupyter Notebooks, RStudio, JetBrains, Vim, Eclipse, Docker, AWS, High Performance Computing (HPC)
\end{rSection}
\vspace{-6pt}
%----------------------------------------------------------------------------------------
% Below chunk could act as concise misc section
%\begin{rSection}{}
%\begin{tabular}{ @{} >{\textit}l @{\hspace{6ex}} l }
%\textbf{Clubs and Organizations:} Club Volleyball, Sigma Phi Epsilon Fraternity, Putnam Club\\
%\textbf{Awards:} ARML 1st place team (B division), 3x AIME qualifier, 3x UNC Math Contest top %10 individual finisher
%\end{rSection}
%\vspace{-6pt}



\begin{rSection}{Activities and Awards} 

\textbf{Awards:} TIAA C-MAPP Scholar, 3x AIME qualifier, ARML 1st place team, 3x UNC Math Contest top 10
\vspace{-4pt}

\textbf{Activities:} Colorado School of Mines - Club Volleyball, Sigma Phi Epsilon Fraternity, Putnam Club
\vspace{-4pt}

\textbf{Outside Interests:} Skiing, Basketball, Hiking, Watching Sports, Solving Math Problems

%\textbf{Colorado School of Mines Club Volleyball} \hfill Aug 2021 - Present
%\vspace{-6pt}
% \begin{itemize}
%    \itemsep -6pt {} 
%     \item Compete in multiple nation-wide tournaments each year as an opposite-hitter; assist with fundraising events
% \end{itemize}
%\vspace{-4pt}

%\textbf{Sigma Phi Epsilon Fraternity} \hfill Sep 2021 - Present
%\vspace{-6pt}
% \begin{itemize}
%    \itemsep -6pt {} 
%     \item Host study sessions for members as a member of the Learning Community cabinet
%    \item Responsible for upholding house budget for supplies as a member of the Finance cabinet
% \end{itemize}
% \vspace{-4pt}
 
%\textbf{Math Club} \hfill Aug 2017 - Present
%\vspace{-6pt}
 %\begin{itemize}
    %\itemsep -6pt {} 
     %\item Meet with club throughout the year to discuss problems and prepare for the Putnam exam
     %\item In June of 2021, I was one of fifteen members on Colorado's ARML team which won 1st %place
 %\end{itemize}
%\vspace{-4pt}

\vspace{-4pt}

\end{rSection}


\end{document}
